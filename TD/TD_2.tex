\documentclass{td_um}
\makeatletter
%--------------------------------------------------------------------------------

%\usepackage[french]{babel}
\usepackage[a4paper,hmargin=20mm,vmargin=25mm]{geometry}
\usepackage{dsfont}
\usepackage[utf8]{inputenc}
\usepackage[T1]{fontenc}
\usepackage{lmodern}

%-------------------------------- AMS --------------------------------------%
\usepackage{amsmath}
\usepackage{amsbsy}
\usepackage{amsfonts}
\usepackage{amssymb}
\usepackage{amscd}
\usepackage{amsthm}


\theoremstyle{definition}
%\newtheorem{defi}{Definition}[section]
\newtheorem{defi}{Definition}[]
\newtheorem{remark}[defi]{Remarque}
\newtheorem{prop}[]{Proposition}
\newtheorem{theo}[]{Th\'eor\`eme}



\usepackage{xcolor}
\usepackage{mathtools}
\usepackage{eurosym}
\usepackage{nicefrac}
\usepackage{latexsym}

\usepackage{multicol}
\usepackage[inline]{enumitem}
\setlist{nosep}
\setlist[itemize,1]{,label=$-$}

\usepackage{sectsty}
%\sectionfont{}
\allsectionsfont{\normalfont\sffamily\bfseries\normalsize}

\usepackage{graphicx}
\usepackage{tikz}
\usetikzlibrary{calc,fadings,decorations.pathreplacing,matrix,arrows,decorations.text}
 \usetikzlibrary{patterns}

\newcommand\chideux[1]{#1<=0||(#1!=int(#1))?1/0:x<=0?0.0:exp((0.5*#1-1.0)*log(x)-0.5*x-lgamma(0.5*#1)-#1*0.5*log(2))}
\newcommand\gauss[2]{1/(#2*sqrt(2*pi))*exp(-((x-#1)^2)/(2*#2^2))} 
\newcommand\student[1]{gamma(.5*(#1+1))/(sqrt(#1*pi)*gamma(.5*#1))*((1+x^2/#1)^(-.5*(#1+1)))}

\usepackage[framemethod=tikz]{mdframed}
\mdfdefinestyle{commandline}{leftmargin=.07\linewidth, rightmargin=.07\linewidth,backgroundcolor=gray!20,linewidth=0pt}
\mdfdefinestyle{codeline}{leftmargin=.01\linewidth, rightmargin=.01\linewidth,linewidth=1pt,backgroundcolor=gray!20,linewidth=0pt}


\usepackage{textcomp}
\usepackage{listings}
\lstloadlanguages{R}
\lstset{upquote=true,
	%columns=flexible,
	keepspaces=true,
	breaklines=true,
	breakindent=0pt,
	basicstyle={\small\ttfamily}
	}
\lstset{extendedchars=true,
	literate={é}{{\'e}}1
		 {è}{{\`e}}1
		 {à}{{\`a}}1
		 {ç}{{\c{c}}}1
		 {œ}{{\oe}}1
		 {ù}{{\`u}}1
		 {É}{{\'E}}1
		 {È}{{\`E}}1
		 {À}{{\`A}}1
		 {Ç}{{\c{C}}}1
		 {Œ}{{\OE}}1
		 {Ê}{{\^E}}1
		 {ê}{{\^e}}1
		 {î}{{\^i}}1
		 {ô}{{\^o}}1
		 {û}{{\^u}}1
		 {°}{{\degre}}1
	}

\lstnewenvironment{script}{%
\lstset{language=,
	    aboveskip=0pt,
	belowskip=0pt
	}
}{}
\surroundwithmdframed[style=commandline]{script}

\lstdefinestyle{codeR}{%
	aboveskip=0pt,
	belowskip=0pt,
	%frame=single,
	commentstyle=\color{green!50!black}\itshape,
	language=R,
	keywordstyle=\color{violet!75},
	stringstyle=\color{red},
	showstringspaces=false,
      %  emph={output,input}, emphstyle=\color{brown!75},
      %  emph={maFonction}, emphstyle=\underline,
	}

\lstnewenvironment{codeR}[1][]{\lstset{style=codeR,#1}}{}
\surroundwithmdframed[style=codeline]{codeR}


\usepackage{algorithm}
\usepackage{algpseudocode}


\usepackage{pgfplots}
\usepgfplotslibrary{fillbetween}
\pgfplotsset{compat=newest}
%\usepgfplotslibrary{external} 
%\tikzexternalize[prefix=./output_latex/]
%\DeclareSymbolFont{RalphSmithFonts}{U}{rsfs}{m}{n}
%\DeclareSymbolFontAlphabet{\mathscr}{RalphSmithFonts}
%\def\mathcal#1{{\mathscr #1}}



% ------------------------------------------- Command -------------------------------%
\providecommand{\abs}[1]{\left|#1\right|}
\providecommand{\norm}[1]{\left\Vert#1\right\Vert}
\providecommand{\U}{\mathcal{U}}
\providecommand{\R}{\mathbb{R}}
\providecommand{\Cc}{\mathcal{C}}
\providecommand{\reg}[1]{\mathcal{C}^{#1}}
\providecommand{\1}{\mathds{1}}
\providecommand{\N}{\mathbb{N}}
\providecommand{\Z}{\mathbb{Z}}
\providecommand{\C}{\mathbb{C}}
\providecommand{\F}{\mathbb{F}}
\providecommand{\K}{\mathbb{K}}
\providecommand{\p}{\partial}
\providecommand{\gR}{\textsc{R}}
\providecommand{\one}{\mathds{1}}
%\renewcommand{\E}{\mathbb{E}}
\renewcommand{\P}{\mathbb{P}}
\providecommand{\ie}{\textit{i.e. }}

\newcommand\rst[2]{{#1}_{\restriction_{#2}}}
%Operateur
\providecommand{\abs}[1]{\left\lvert#1\right\rvert}
\providecommand{\sabs}[1]{\lvert#1\rvert}
\providecommand{\babs}[1]{\bigg\lvert#1\bigg\rvert}
\providecommand{\norm}[1]{\left\lVert#1\right\rVert}
\providecommand{\bnorm}[1]{\bigg\lVert#1\bigg\rVert}
\providecommand{\snorm}[1]{\lVert#1\rVert}
\providecommand{\prs}[1]{\left\langle #1\right\rangle}
\providecommand{\sprs}[1]{\langle #1\rangle}
\providecommand{\bprs}[1]{\bigg\langle #1\bigg\rangle}

\DeclareMathOperator{\deet}{Det}
\DeclareMathOperator{\supp}{Supp}
\DeclareMathOperator{\sinc}{sinc}


\usepackage{ifthen}

\newcommand{\eno}[1]{%
	\ifthenelse{\equal{\version}{eno}}{#1}{}%
}
\newcommand{\cor}[1]{%
        \ifthenelse{\equal{\version}{cor}}{
\medskip 

{\small \color{gray} #1}

\medskip 
}{}
}

%------------------------------------------------------------------------------
\DeclareUnicodeCharacter{00A0}{~}
\makeatother



\def\version{eno}
%\def\version{cor}

\usepackage{hyperref}
\ue{HAX814X}

\providecommand{\T}{\mathbb{T}}
\providecommand{\1}{\mathds{1}}
\title{TD 2 : Régression linéaire multiple}


\newcommand{\miniscule}{\@setfontsize\miniscule{5}{6}}
%-----------------------------------------------------------------------------
\begin{document}
	\maketitle
	
	
	\exo{Rappels de cours}
	
	\begin{enumerate}
		\item Rappeler le principe d'une régression linéaire multiple. Préciser les hypothèses.
		\item Faire un schéma pour donner une interprétation géométrique à la régression linéaire multiple.
		\item Donner l'expression de la matrice de projection $\mathbf{P}^{\mathbf{X}}$ et de l'estimateur $\hat \beta$. Vérifier que $\mathbf{P}^{\mathbf{X}}$ est bien une matrice de projection.
		\item Quelles sont les hypothèses supplémentaires dans le cas gaussien ?
	\end{enumerate}
	\textit{On conseille de toujours faire attention à la dimension des objets (matrices et vecteurs) qu'on manipule.}
	
	
	\exo{Régression simple vs régression multiple}
	\begin{enumerate}
		\item Rappeler les expressions de $\hat \beta_0$ et $\hat \beta_1$ dans le cas d'une régression simple.
		\item Rappeler l'expression de $\hat \beta$ dans le cas d'une régression multiple.
		\item Retrouver le résultat de la question 1. à partir de celui de la question 2.
		\item Rappeler les expressions des variances et covariance de $\hat \beta_0$ et $\hat \beta_1$ dans le cas d'une régression simple.
		\item Rappeler l'expression de la matrice de variance-covariance de $\hat \beta$ dans le cas d'une régression multiple.
		\item Retrouver le résultat de la question 4. à partir de celui de la question 5.
	\end{enumerate}

	\cor{\newpage}

	\exo{} On étudie l'évolution d'une variable $y$ en fonction de deux variables $x$ et $z$. On dispose de $n$ observations de ces variables. On note $X=\begin{pmatrix}\1 & x & z\end{pmatrix}$, où $\1$ est le vecteur constant, et $x$ et $z$ sont les vecteurs des variables explicatives.
		\begin{enumerate}
			\item Nous avons obtenu les résultats suivants:
			\[
			X^T X=\begin{pmatrix}
				30 & 0 & 0 \\
				? & 10 & 7 \\
				? & ? & 15
			\end{pmatrix}\,.
			\]
			\begin{enumerate}
				\item Donner les valeurs manquantes. Que vaut $n$ ?
				\item Calculer le coefficient de corrélation empirique entre $x$ et $z$.
			\end{enumerate}
			\item La régression linéaire de $Y$ sur $\1 , x, z$ donne
			\[
			y=-2 \1 + x + 2 z + \hat{\varepsilon}\,, \quad S C R = \|\hat{\varepsilon}\|^{2} = 12\,.
			\]
			\begin{enumerate}
				\item Calculer $\sum_{i=1}^n \hat \varepsilon_i$, puis en déduire la valeur de la moyenne arithmétique $\bar{y}$.
				\item Calculer la somme des carrés expliquée (SCE), la somme des carrés totale (SCT) et le coefficient de détermination $R^{2}$. %que l'on notera $R_{\1,x,y}$.
			\end{enumerate}
		\item 
		\begin{enumerate}
			\item Calculer $X^T y$ en utilisant la valeur de $\hat{\beta}$, puis en déduire $\sum x_{i} y_{i}$ et $\sum z_{i} y_{i}$.
			\item Calculer les coefficients de corrélation $\rho_{x,y}$ et $\rho_{z,y}$. En déduire la valeur du $R^{2}$ pour le modèle de régression de $y$ par $\1 $ et $x$, puis de $y$ par $\1 $ et $z$.% On les note $R^2_{\1,x}$ et $R^2_{\1,y}$ respectivement.
		\end{enumerate}
	\end{enumerate}

\cor{\newpage}
	
	\exo{}
	\begin{enumerate}
		\item Nous avons une variable $Y$ à expliquer par une variable $X$. Nous avons effectué $n=2$ mesures et trouvé
		\[
		\left(x_{1}, y_{1}\right)=(4,5) \text { et }\left(x_{2}, y_{2}\right)=(1,5)
		\]
		Représenter les variables, estimer $\beta$ dans le modèle $y_{i}=\beta x_{i}+\varepsilon_{i}$ et représenter $\hat{Y}$.
		\item Nous avons maintenant une variable $Y$ à expliquer par deux variables $X_{1}$ et $X_{2}$. Nous avons effectué $n=3$ mesures et trouvé
		\[
		(x_{1,1}, x_{1,2},y_{1})=(3,2,0),\,\quad
		(x_{2,1}, x_{2,2}, y_{2})=(3,3,5),\,\quad
		(x_{3,1}, x_{3,2}, y_{3})=(0,0,3)\,.
		\]
		Représenter les variables, estimer $\beta$ dans le modèle $y_{i}=\beta_{1} x_{i, 1}+\beta_{2} x_{i, 2}+\varepsilon_{i}$ et représenter $\hat{Y}$.
	\end{enumerate}
	
	\cor{\newpage}
	
	\exo{} Soit $X$ une matrice de taille $n \times p$ composée de $p$ vecteurs indépendants de $\mathbb{R}^{n}$. Nous notons $Z$ la matrice composée des $q < p$ premiers vecteurs de $X$. Nous avons les deux modèles suivants :
	\[
	(1) \quad Y=X \beta+\varepsilon \quad \text { et } \quad (2) \quad Y=Z \tilde \beta+\psi\,.
	\]
	Comparer les $R^{2}$ dans les deux modèles.

\cor{\newpage}

	\exo{}  Soit $\theta_{1}$ et $\theta_{2}$ deux paramètres réels inconnus et soit :
	\begin{itemize}
			\item $Y_{1}$ un estimateur sans biais de $\theta_{1} + \theta_{2}$ et de variance $\sigma^{2}$
			\item $Y_{2}$ un estimateur sans biais de $2 \theta_{1} - \theta_{2}$ et de variance $4 \sigma^{2}$ 
			\item $Y_{3}$ un estimateur sans biais de $6 \theta_{1} + 3 \theta_{2}$ et de variance $9 \sigma^{2}$ 	
	\end{itemize}
		Les estimateurs $Y_{1}$, $Y_{2}$ et $Y_{3}$ étant indépendants, nous cherchons les estimateurs sans biais de $\theta_{1}$ et $\theta_{2}$, linéaires en $Y_{1}$, $Y_{2}$ et $Y_{3},$ et de variance minimale.
	\begin{enumerate}
		\item Notons $\tilde{\theta} = \alpha Y_{1} + \beta Y_{2} + \gamma Y_{3}$.
		\begin{enumerate}
			\item Quelles sont les équations à satisfaire pour que $\tilde{\theta}$ soit un estimateur sans biais de $\theta_{1} ?$
			\item Dans ce cas-là, exprimer la variance de $\tilde{\theta}$ et la minimiser.
			\item Idem pour $\theta_{2}$.
		\end{enumerate}
		\item On pose $Z_{1} = Y_{1}$, $Z_{2} = Y_{2} / 2$, et $Z_{3} = Y_{3} / 3$, et on note $Z=\left(Z_{1}, Z_{2}, Z_{3}\right)^T$ et $\theta=\left(\theta_{1}, \theta_{2}\right)^T$.
		\begin{enumerate}
			\item Trouver la matrice $X$ telle que $\mathbb{E}[Z]=X \theta$.
			\item Que vaut la matrice de variance-covariance de $Z$ ?
			\item On peut alors écrire $Z=X \theta+\varepsilon$. Retrouver les estimateurs de $\theta_{1}$ et $\theta_{2}$ calculés question 1.
		\end{enumerate}
	\end{enumerate}
	
	\cor{\newpage}
	
	\exo{Régression sur données agrégées par groupes} On suppose le modèle de régression
	\[
	Y = X \beta + \varepsilon\,,
	\quad \text{avec} \quad
	\mathbb{E}[\varepsilon]=0
	\quad \text{et} \quad \text{Var}(\varepsilon)=\sigma^{2} I_{n}\,.
	\]
	Les données individuelles $\left(x_{i 1}, \ldots, x_{i p}, y_{i}\right)$ ne sont cependant pas disponibles. On observe seulement les moyennes sur $I$ groupes, notés $C_{1}, \ldots, C_{I},$ d'effectifs $n_{1}, \ldots, n_{I}$ :
	\[
	\bar{y}_{k}=\frac{1}{n_{k}} \sum_{i \in C_{k}} y_{i}
	\quad \text { et } \quad
	\bar{x}_{k j}=\frac{1}{n_{k}} \sum_{i \in C_{k}} x_{i j}\,.
	\]
	En notant $\bar{\varepsilon}_{k}=\frac{1}{n_{k}} \sum_{i \in C_{k}} \varepsilon_{i},$ on a alors $\bar{Y}=\bar{X} \beta+\bar{\varepsilon}$.
	\begin{enumerate}
		\item Calculer $\mathbb{E}[\bar{\varepsilon}]$ et $\text{Var}(\bar{\varepsilon})$.
		\item On pose
		\[
		M = \operatorname{diag}(\sqrt{n_{1}}, \ldots, \sqrt{n_{I}})\,,
		\quad
		Y^{*} = M \bar{Y}\,,
		\quad
		X^{*} = M \bar{X}\,,
		\quad
		\varepsilon^{*}=M \bar{\varepsilon}\,.
		\]
		Quelle est la relation entre $Y^{*}$, $X^{*}$ et $\varepsilon^{*}$ ? Calculer $\mathbb{E}[\varepsilon^{*}]$ et $\text{Var}(\varepsilon^{*})$.
		\item En déduire un estimateur de $\beta$.
		\item Application numérique $: I=3$ avec $n_{1}=1$ et $n_{2}=n_{3}=2$. $\bar{X}_{1}^T=(1,1,1), \bar{X}_{2}^T=(7,12,5)$ et $\bar{Y}^T=(15,25,10)$.
	\end{enumerate}
\end{document}


